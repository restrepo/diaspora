
\section{Trabajo colaborativo}
\begin{evaluacion}
  TOTAL: 20\% 
  * Trayectoria de colaboración previa entre el colombiano
  en el exterior y el grupo de investigación en Colombia.
  * Vinculación de diversos actores (grupos de investigación) para
  trabajar conjuntamente con la disapora científica en Colombia
  * Estableciemiento de alianzas estrátegicas que existan formalmente
  la institución en Colombia e instituciones de otros paises
  diferentes al de Colombia.
\end{evaluacion}

\subsection{Antecedentes}
\begin{instrucciones}
  En caso de que exista, describa con un máximo de 200 palabras la
  trayectoria de colaboración y alianzas estratégicas formales entre
  el colombiano o institución a la que éste está vinculado en el
  exterior y la entidad ejecutora del proyecto.  
\end{instrucciones}

\subsection{Red de trabajo}
\begin{instrucciones}
  Describa los grupos de investigación o instituciones vinculadas al
  proyecto (dentro o fuera del país), diferentes al colombiano enlace
  en el exterior y a la entidad ejecutora en Colombia. Incluya el rol
  que desempeña cada actor dentro del proyecto.
\end{instrucciones}

\begin{tabular}{|l|l|}\hline
Actor & Responsabilidad\\\hline  
Grupo de Fenomenología de Interacciones Fundamentales & Inv. Ppal,
estudiantes\\\hline
& \\\hline
\end{tabular}

\subsection{Alianzas estratégicas}
\begin{instrucciones}
  En caso de que el proyecto implique la realización de una alianza
  estratégica con una entidad en el exterior, por favor describa
  brevemente en qué consiste y si ya hay algún avance.
\end{instrucciones}

%%% Local Variables: 
%%% mode: latex
%%% TeX-master: "proyecto"
%%% End: 
