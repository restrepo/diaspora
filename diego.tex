\section{Información sobre el colombiano en el exterior}
\begin{evaluacion}
  TOTAL: 15\%
  * Excelencia académica, producción investigativa y trayectoria profesional
  * Experiencia específica en el tema del proyecto
\end{evaluacion}
\subsection{Nombres y apellidos:}
Diego Aristizábal Sierra.
\subsection{Lugar y fecha de nacimiento:}
Medellin, 1973
\subsection{Correo electrónico:}
\texttt{daristi@gmail.com}
\subsection{País y ciudad de residencia:}
Bélgica, Liège
\subsection{Formación académica:}
\begin{itemize}
\item Físico, Universidad de Antioquia (Colombia) (2000).
\item Magíster en Física, Universidad de Antioquia (Colombia) (2003).
\item Ph. D., Universidad de Valencia (España) (2007).
\end{itemize}
\subsection{Producción intelectual de los tres últimos años}
\subsubsection{Artículos publicados en revistas indexadas:}
  D.~Aristizabal Sierra, C.~E.~Yaguna,
  %``On the importance of the 1-loop finite corrections to seesaw neutrino masses,''
  JHEP {\bf 1108 } (2011)  013.

  D.~Aristizabal Sierra, J.~F.~Kamenik, M.~Nemevsek,
  %``Implications of Flavor Dynamics for Fermion Triplet Leptogenesis,''
  JHEP {\bf 1010 } (2010)  036.

  D.~Aristizabal Sierra, F.~Bazzocchi, I.~de Medeiros Varzielas, L.~Merlo, S.~Morisi,
  %``Tri-Bimaximal Lepton Mixing and Leptogenesis,''
  Nucl.\ Phys.\  {\bf B827 } (2010)  34-58.

  D.~Aristizabal Sierra, D.~Restrepo, O.~Zapata,
  %``Decaying Neutralino Dark Matter in Anomalous U(1)(H) Models,''
  Phys.\ Rev.\  {\bf D80 } (2009)  055010.

  D.~Aristizabal Sierra, M.~Losada, E.~Nardi,
  %``Lepton Flavor Equilibration and Leptogenesis,''
  JCAP {\bf 0912 } (2009)  015.

  D.~Aristizabal Sierra, L.~A.~Munoz, E.~Nardi,
  %``Implications of an additional scale on leptogenesis,''
  J.\ Phys.\ Conf.\ Ser.\  {\bf 171 } (2009)  012078.

  D.~Aristizabal Sierra, L.~A.~Munoz, E.~Nardi,
  %``Purely Flavored Leptogenesis,''
  Phys.\ Rev.\  {\bf D80 } (2009)  016007.

  D.~Aristizabal Sierra, J.~Kubo, D.~Restrepo, D.~Suematsu, O.~Zapata,
  %``Radiative seesaw: Warm dark matter, collider and lepton flavour violating signals,''
  Phys.\ Rev.\  {\bf D79 } (2009)  013011.
\subsubsection{Libros y capítulos publicados:}
\subsection{Reconocimiento por excelencia académica:}
%maximo 200 palabras
\subsection{Experiencia específica en el tema del proyecto:}
%maximo 200 palabras
\subsection{Trayectoria laboral:}
\begin{instrucciones}
  Liste su experiencia laboral en orden cronológico (empezando por lo
  actual) organizándolos cada referencia por los siguientes criterios:
  nombre de la empresa o entidad, pública o privada, país, cargo,
  fecha de ingreso y fecha de retiro.
\end{instrucciones}
\begin{itemize}
\item  Institut de physique, Universite de Liege, entidad pública, Bélgica, Postdoc, 2010-2012. 
\end{itemize}
\subsection{Tiempo total de experiencia:}
\begin{instrucciones}
  Indique el tiempo total de su experiencia laboral en número de años
  y meses (ejemplo: 5 años, 3 meses).
\end{instrucciones}


%%% Local Variables: 
%%% mode: latex
%%% TeX-master: "proyecto"
%%% End: 
