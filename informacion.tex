
\section{Información sobre el proyecto}

\subsection{Sector(es)  en el que se desarrolla el proyecto:}


\subsection{Título:                                        }
\begin{instrucciones}
  El sector puede ser alguno de los propuestos en la convocatoria
  (agricultura, energía, agua, biodiversidad, desarrollo tecnológico e
  innovación) u otros.
\end{instrucciones}
Ciencias Básicas
\subsection{Resumen ejecutivo:                            }
%Máximo 200 palabras.
\subsection{Monto económico total (incluida contrapartida):}
\subsection{Antecedentes:                                  }
Los avances recientes en física de partículas y cosmología han dado
lugar a un entendimiento claro de las tres fronteras a lo largo de la
cual la física de partículas debe avanzar para resolver algunos de los
misterios cruciales de nuestro universo: tales como el origen de la
masa, la naturaleza de la materia oscura y la energía oscura, la
generación de la asimetría materia-antimateria, y la posible
unificación de las fuerzas. Las tres fronteras que se ilustran en la
Figura~\ref{fig:1}, se han identificado como la Frontera de Energía,
la Frontera de Intensidad, y la Frontera Cósmica \cite{fermilab}. Este
proyecto cubrirá las tres frentes y conectará física de partículas y
cosmología. En el proyecto se estudiarán varios modelos teóricos
confrontándolos con los resultados experimentales recientes y haciendo
predicciones para los experimentos en marcha y los que entrarán
próximamente en funcionamiento.

\begin{figure}
  \centering
\includegraphics[scale=0.3]{three-frontiers-large}
  \caption{Fronteras de energía. From \cite{fermilab}}
  \label{fig:1}
\end{figure}

Nos encontramos ahora en una época de efervescencia experimental con
detectores de partículas instalados desde las alturas de satélites
artificiales, hasta las profundidades de laboratorios subterráneos a
kilómetros de profundidad. Esto marca una época excitante para la
física de partículas con nuevos datos experimentales disponibles en
las tres fronteras antes mencionadas. La posible correlación de datos
experimentales entre varias de las fronteras podrían permitir un
entendimiento más profundo de los constituyentes del Universo.

La \emph{Frontera de Energía}, ha alcanzado la escala del Tera, la
energía a la cual se rompe la simetría electrodébil con la puesta en
funcionamiento del Large Hadron Collider (LHC). Ubicado en un túnel a
100 metros de profundidad y con una circunferencia de 27 Km, el LHC
posee cuatro detectores, dos de los cuales: ATLAS y CMS, están
especialmente diseñados para encontrar señales de nueva física. El LHC
ha comenzado a operar en el 2010, aunque hasta el 2012 lo hará a la
mitad de la energía para la cual fue diseñado. A partir del 2014
aproximadamente, comenzará a funcionar a la energía de diseño de
14~TeV. Para finales del 2012 el LHC habrá completado su primera fase
de operación a una energía de centro de masa de 7~TeV y habrá
acumulado al menos 8/fb de datos. Con esta luminosidad se podrá
vislumbrar una escala de energía que hasta ahora no había sido
explorada. La prioridad es la búsqueda del Higgs del Modelo Estándar,
el cual a la fecha aún no ha sido encontrado. Pero para comienzos del
2011 se espera obtener la primeras evidencias de su existencia o
excluirlo hasta un rango de masa de unos 600~GeV.

Quizás el resultado experimental más importante en los últimos años de
necesidad de física más allá del modelo estándar (ME) de las
partículas elementales es el descubrimiento de que los neutrinos
tienen masa, las cuales han resultado ser pequeñas aunque diferentes de
cero. Las diferencias de masa al cuadrado de neutrinos, además de sus
correspondientes ángulos de mezcla, son necesarios para poder explicar
las observaciones de oscilaciones de neutrinos a medida que se
propagan sobre grandes distancias. Debido a que los neutrinos
interactúan sólo débilmente, los experimentos de neutrinos requieren
de detectores fabricados de materiales masivos y rayos muy
intensos. Los experimentos de neutrinos exploran de ésta manera la
\emph{Frontera de Intensidad}. Los experimentos en esta frontera se
enfocan ahora en estudios más precisos de oscilaciones de neutrinos
así como en búsqueda de nuevas fuentes de violación de CP, mezcla de
sabores de leptones cargados, decaimientos raros, y en la
determinación de la velocidad de propagación de neutrinos
energéticos. Los experimentos que usan rayos muy intensos a energías
inferiores que el LHC pueden proveer información complementaria a los
posibles descubrimientos de los detectores ATLAS y CMS. Un decaimiento
raro que proviene del intercambio de una partícula de masa alta puede
contener información sobre las propiedades del estado intercambiado
aunque éste sea demasiado pesado para ser producido directamente.

Recientemente se ha venido acumulando evidencia experimental respecto
al ángulo de mezcla del sector de neutrinos que aún falta por
determinar, mostrando que dicho ángulo no sólo es diferente de cero
sino que puede ser suficientemente grande como para permitir violación
de CP en el sector leptónico \cite{valle}. La violación de CP en el
sector leptónico es un ingrediente necesario para generar la asimetría
de materia-antimateria a través de
leptogénesis~\cite{Davidson:2008bu}.


La \emph{Frontera Cósmica} utiliza laboratorios subterráneos,
telescopios basados en tierra y telescopios instalados en satélites
para explorar la componentes oscuras de materia y energía, las huellas
de la inflación y el origen y destino del universo. Las observaciones
de la Frontera Cósmica han alcanzado una precisión mucho mayor de la
podría haber sido imaginada dos décadas antes. Estos han conseguido
determinar detalles del universo primitivo los cuales son cada vez más
consistentes con el ``Modelo Estándar'' de Cosmología basado en la
constante cosmológica y en la materia oscura fría ($\Lambda$CDM), que
dan cuenta del 95\% del contenido energético del Universo. Técnicas
novedosas como lentes gravitacionales, han aportado significativamente
a nuestro conocimiento del pasado cosmológico, en particular en
aumentar la evidencia experimental de que la materia oscura del
universo está compuesta de partículas masivas débilmente
interactuantes (WIMPS), formando halos de materia alrededor de las
galaxias. Cómo los neutrinos constituyen sólo alrededor del 0.5\% del
contenido energético del universo, los WIMPS deben corresponder a
nuevas partículas no presentes en el ME, constituyéndose en la segunda
evidencia experimental de necesidad de física más allá del ME. Estas
partículas deben ser o bien estables, o inestables pero con un tiempo
de vida mayor que la edad del universo, generando productos de
aniquilación en el primer caso, o de decaimiento en el segundo
dando lugar a rayos cósmicos que pueden llegar a los detectores instalados
en satélites artificiales orbitando la tierra.  El valor de múltiples
estudios con detectores de diversos tipos de partículas y radiación
electromagnética sobre un rango muy amplio (incluyendo rayos gamma) se
hace evidente en el conocimiento detallado que se ha logrado alcanzar
y el que se espera mejorar con los experimentos que recientemente han
entrado en funcionamiento.

%comentar sobre una simetría Z_2
La materia oscura representa un gran desafío teórico y experimental,
tanto para la astrofísica moderna como para la física de partículas
\cite{Bertone:2004pz, Amsler:2008zzb, Bertone:2010,Jungman:1995df}. En
la actualidad no sabemos cuál es su verdadera identidad y no se tiene
ninguna explicación en la teoría que rigen las interacciones de las
partículas fundamentales, el Modelo Estándar.

Muchos esfuerzos teóricos se han realizado para construir teorías más
allá del Modelo Estándar con candidatos prometedores de materia
oscura. Entre los candidatos a ser una partícula masiva débilmente
interactuante \cite{Bertone:2004pz,Jungman:1995df} (no bariónica, lo
suficientemente estable y neutra eléctricamente), tenemos la más
ligera de las partículas supersimétricas, escalares neutros
adicionales, neutrinos derechos, y las partículas de Kaluza-Klein,
cada uno con diferentes implicaciones experimentales.


%%%%%%%%%%
Desde el 2008 varios experimentos sobre rayos cósmicos como ATIC
\cite{:2008zzr} y los satélites PAMELA \cite{Adriani:2008zr} y
Fermi--LAT \cite{Abdo:2009zk}, han venido reportando un exceso en el
flujo de electrones y positrones en rayos cósmicos. 
%
Los resultados experimentales muestran un exceso
inesperado en comparación con el background convencional, tanto en el
flujo de electrones más positrones como en la fracción de positrones,
señalando la existencia de una fuente adicional de electrones y
positrones en el halo de la Vía Láctea, mientras que los datos de
antiprotones están de acuerdo con el background astrofísico esperado. 
%
Estos resultados han dado lugar a un sinnúmero de publicaciones
tratando de explicar su origen. Las medidas cada vez más precisas de
rayos gamma por parte de Fermi--LAT, pueden ayudar a discernir si el
origen de las anomalías detectadas en electrones y positrones es
debida a fuentes astrofísicas como pulsares cercanos, o a la
aniquilación o el decaimiento de materia oscura.
%
El detector de rayos cósmicos AMS-02~\cite{ams:2009},
ha sido instalado recientemente en la estación espacial internacional
para medir el espectro de antiprotones y positrones en un rango de
energía mucho más amplio y con una estadística mucho mejor que PAMELA.

Los experimentos de detección directa de materia oscura instalados en
laboratorios subterráneos como XENON100 \cite{Aprile:2011ts}, han
comenzado a explorar las regiones predichas por algunos de los modelos
más estudiados de materia oscura. Para los próximos años se espera
cubrir todo el espacio de parámetros para materia oscura del modelo
estándar supersimétrico mínimo restringido (Constrained MSSM de sus
siglas en Inglés).  



\subsection{Justificación:                                 }
\begin{instrucciones}
  CODI: 

  * ¿Está bien definido el problema que se quiere investigar?:
  Fenomenología de modelos más allá del ME motivados por evidencias
  fenomenológicas del ME.  
  * ¿Es clara su justificación desde el punto de vista académico,
  científico, tecnológico, social, económico y legal? (15).
  científico: contribuir a la correspondiente rama del conocimiento

  COLCI: en este ítem usted deberá describir de forma precisa y completa la
  
  * naturaleza y magnitud del problema de investigación que se quiere
  abordar:
  ** Construir modelos nuevos.
  ** Explorar modelos existentes.
  Formule claramente las preguntas concretas a las cuales se
  quiere responder en el contexto del problema planteado.
\end{instrucciones}
%Tesis
El Grupo de Fenomenología de Interacciones Fundamentales de la
Universidad de Antioquia, en colaboración con Grupos emergentes en
otras instituciones la región conformados por egresados de doctorado
de nuestro Grupo, se ha enfocado en la investigación científica de
aspectos fenomenológicos en estas tres fronteras de la física de altas
energías y la cosmología. Estas investigaciones se han hecho en
colaboración con investigadores internacionales, especialmente con
científicos colombianos trabajando en el exterior. A través de este
proyecto se busca dar continuidad a estos desarrollos a través de
investigaciones de impacto científico en la comunidad científica de
éstas áreas de frontera. En éste proyecto se explorarán diferentes
extensiones del modelo estándar que explican estos datos y tienen
predicciones concretas para los experimentos presentes y futuros en
éstas tres fronteras de la física de partículas y la cosmología.

%pequeño resumen de lo que se ha hecho

%cosas por hacer
%1. Estudio sistemático de gravitino 

Con respecto a la detección, cuando la simetría paridad R se conserva, los gravitinos tienen interacciones muy débiles y por lo tanto ninguna señal de la materia oscura gravitino se puede observar en los experimentos de detección directa o indirecta. En cuanto a las señales en los colisionadores, la producción directa de gravitinos es muy suprimida, pero la segunda partícula supersimétrica más  ligera puede dejar rastros en los detectores. Por otro lado, si el gravitino es inestable (debido a la violación de paridad R) sus productos de desintegración puede llevar a  señales observables en las búsquedas indirectas de materia oscura \cite{Bertone:2007aw,Ibarra:2007wg,Covi:2008jy,Ibarra:2008qg}. Los gravitinos puede ser detectados indirectamente a través del decaimiento a rayos gamma, neutrinos o antimateria. De particular interés se tiene la región en la cual la masa del gravitino es menor que 10 GeV, ya que el gravitino puede decaer únicamente a un neutrino y un rayo gamma, produciendo lo que se conoce como una línea de rayos gamma (un fotón monoenergético). Los datos del espectro de rayos gamma extragaláctico, tomados por los satélite EGRET y FERMI, ponen fuertes cotas sobre el espacio de parámetros de estos modelos. De hecho, en la referencia \cite{Yuksel:2007dr} se obtuvieron restricciones sobre el tiempo de vida del candidato a materia oscura en los modelos supersimétricos con violacion de paridad R, para un rango de masas de $10^{-5}$ a $10$ GeV. En esta misma direcci\'on, en la referencia \cite{Vertongen:2011mu} se realizó una búsqueda sistemática de señales de líneas de rayos gamma en dichos modelos, sin encontrar evidencia de éstas. Con este resultado, se obtuvieron restricciones sobre el tiempo de vida del candidato a materia oscura para una masas entre $1<m_{\tilde G}< 500$ GeV. 

%
Supersimetría \cite{Martin:1997ns,Haber:1984rc} es una de las teorías propuestas para resolver el problema de la jerarquía en el Modelo Estándar. Además de tener la virtud de unificar a los acoplamientos gauge del Modelo Estándar, la supersimetría a bajas energías proporciona un candidato a materia oscura que corresponde a la partícula supersimétrica más ligera, la cual es estable cuando se supone la conservación de la simetría de paridad R \cite{Ellis:1983ew}. La conservación de la simetría de paridad R prohibe la aparición de operadores que violan el número bariónico y leptónico, evitando así el decaimiento prematuro del protón. Cuando paridad R no se conserva en el Modelo Estándar supersimétrico \cite{Barbier:2004ez}, estos operadores pueden conducir al decaimiento de la partícula supersimétrica más ligera, teniendo profundas implicaciones en la búsqueda de supersimetría en aceleradores  y en los experimentos de detección directa  e indirecta. Cabe mencionar que el programa de detección de materia oscura se ha centrado en la búsqueda de sus señales en los aceleradores como el Gran Colisionador de Hadrones \cite{Baltz:2006fm,Cho:2008tj,Nath:2010zj}, en la detección de la energía de retroceso en la dispersión de las partículas de materia oscura con núcleos atómicos \cite{Green:2007rb,Bertone:2007xj,Drees:2008bv,Green:2008rd}, y la detección indirecta a través de los estados finales de la aniquilación y/o decaimiento de la materia oscura \cite{Bertone:2007aw,Eichler:1989br,Arvanitaki:2008hq,Ibarra:2008jk,Ibarra:2008qg,Buckley:2009kw,Ibarra:2009tn,Ruderman:2009ta}, tales como fotones, neutrinos y la antimateria. 

Cuando supersimetría se promueve a ser una simetría local de la naturaleza, la teoría de supergravedad resultante requiere un supermultiplete que incluya el gravitón y su supercompañero, el gravitino \cite{Martin:1997ns,Nilles:1983ge}. El gravitino ($\tilde G$) adquiere masa a partir de la ruptura espontánea de la supersimetría (llamado super mecanismo de Higgs), la cual viene dada por $m_{\tilde G}=<F>/M_P$, donde $<F>$ es el valor esperado de vacío del campo auxiliar que rompe la supersimetría, y $M_P$ es la masa de Planck. Por lo tanto, dependiendo del escenario de rotura de supersimetría, el rango de la masa del gravitino va desde los eV hasta mas allá de la escala del TeV. 

La aparición natural del gravitino en supergravedad trae un problema con el escenario de la cosmología estándar. En el Universo primitivo, cuando el gravitino está en equilibrio térmico con el plasma, la densidad reliquia de gravitinos es mayor que la densidad crítica, lo que implica que el universo podría auto-contraerse. Por lo tanto, es fundamental invocar una fase inflacionaria en el Universo con el objetivo de diluir la densidad reliquia de gravitinos. Sin embargo, los gravitinos también se producen en la fase de recalentamiento (a través del decaimiento del inflatón) después de la inflación. Durante o después de la nucleosíntesis, el decaimiento de la segunda partícula supersimétrica más ligera o el decaimiento del gravitino mismo podrían generar una lluvia electromagnética y de hadrones que echarían a perder la predicción éxitosa de las abundancias de elementos ligeros \cite{Sarkar:1995dd}. La solución a este problema se obtiene mediante la introducción de una violación bilineal de paridad R \cite{Buchmuller:2007ui}, lo cual permite a la segunda partícula supersimétrica más ligera decaer en las partículas del Modelo Estándar antes de la nucleosíntesis, y permite a los restantes gravitinos tener un tiempo de vida mayor que la edad del universo. En este último caso, el gran tiempo de vida que se requiere para un candidato de materia oscura inestable se obtiene gracias a las débiles interacciones del gravitino, las cuales  son suprimidas por la escala de Planck y por los pequeños acoplamientos bilineales que violan paridad R. Un escenario con gravitino como materia oscura y con violación bilineal de paridad R también se ve favorecido por la teoría que explica la asimetría entre materia y antimateria, la leptogénesis, ya que alivia la tensión entre la alta temperatura de recalentamiento requerida por leptogénesis y las restricciones provenientes de la teoría de la Nucleosíntesis. Por lo tanto, en modelos con violación bilineal de paridad, el gravitino como materia oscura producido en el Universo temprano es viable y bien motivado. 

%2. masas y mezclas de neutrinos violación de CP
%3. materia oscura
%4. 
Si la masa de los neutrinos es total o parcialmente explicada por
métodos radiativos, no sólo es posible dar cuenta de la pequeñez de
sus masas con respecto a la de los otros fermiones, sino también,
hacer predicciones muy concretas en aceleradores de partículas, como
el LHC. 


  %evidencia 4
Las observaciones astronómicas sugieren que el Universo está compuesto
en su mayor parte de materia. En el contexto del big-bang, esto
implica que en algún momento grandes cantidades de materia y
antimateria se aniquilaron dejando el pequeño exceso de materia que
constituye el Universo observable actual. El problema de explicar el
exceso inicial de materia sobre antimateria se conoce con el nombre de
bariogénesis. Dentro del modelo estándar, aunque contiene los
ingredientes necesarios, no es posible explicar bariogénesis.

%conclusion
Un modelo ideal sería uno que de cuenta de las masas y mezclas de
neutrinos, tenga un candidato a materia oscura que sirva para explicar
el exceso de positrones en experimentos de rayos cósmicos y a la vez
contenga los ingredientes para explicar bariogénesis.  En este
proyecto pretendemos formular un modelo de éstas características,
además queremos continuar explorando otros posibles modelos que puedan
dar cuenta de alguna de las evidencias fenomenológicas (masas de
neutrinos o materia oscura) que requieren una extensión del modelo
estándar.


\begin{proyecto}
  Con base en lo planteado anteriormente, lo que proponemos en este
  proyecto es tratar de responder la siguiente pregunta: ¿Cuáles
  serían las restricciones impuestas por los resultados experimentales
  presentes y futuros de los experimentos de física de partículas
  sobre modelos que presenten una partícula candidata a materia oscura
  o que generen masas para los neutrinos?
\end{proyecto}
 



Los experimentos de neutrinos solares, atmosféricos y de reactor han demostrado que los neutrinos tienen masa y ángulos de mezcla entre las diferentes generaciones diferentes de cero \cite{Ahmad:2002jz,Fukuda:1998mi,Eguchi:2002dm}. Estos resultados experimentales representan en la actualidad las evidencias más importantes para la física más allá del Modelo Estándar. Hay varias maneras de generar la masa de los neutrinos en modelos más allá del Modelo Estándar. Sin duda el mecanismo más conocido para generar pequeñas masas de neutrinos de Majorana es el mecanismo seesaw \cite{Minkowski:1977sc}, el cual involucra la introducción de neutrinos derechos a una escala de energía muy alta. En modelos supersimétricos con violación bilineal de paridad R también es posible explicar las masas y mezclas de neutrinos \cite{Hirsch:2004he,Hirsch:2008ur}. Por un lado se encuentra la generación a nivel árbol de la escala de masa atmosférica, el ángulo de mezcla atmosférico y el ángulo de reactor. En el otro lado, la escala de masa solar y el ángulo de mezcla solar se obtienen por medio de las correcciones cuánticas a un loop de la matriz de masa de los neutrinos a nivel árbol. %Cabe señalar que cuando los acoplamientos bilineales que violan paridad R son muy pequeños, las correspondientes valores para las masas y mezclas de los neutrinos son negligibles. Sin embargo, se puede aun explicar los datos de física de neutrinos a tráves de la implementación del mecanismo seesaw, agregando neutrinos derechos al contenido de partículas del Modelo Estándar supersimétrico. De hecho, los modelos con violacion bilineal de paridad R con gravitino como materia oscura utilizan el mecanismo seesaw para explicar simultáneamente la física de neutrinos y la asimetría entre materia y antimateria por medio de la leptogénesis. En estos modelos se requiere que los acoplamientos bilineales sean muy pequeños $\xi_i<10^{-7}$ para no borrar la asimetría leptónica, una temperatura de recalentamiento muy grande $T_R>10^9$ GeV y una masa para el gravitino mayor a 10 GeV. De estas restricciones se desprende el hecho que la leptogénesis excluye la posibilidad de generar las masas de los neutrinos por medio de los acoplamientos bilineales. 


%\subsection{Planteamiento del problema}

En el marco de los modelos supersimétricos con violación bilineal de paridad R y materia oscura de gravitinos, se supone que el gravitino es la más ligera de las partículas supersimétricas y que da cuenta de la densidad de materia oscura observada del Universo. En estos modelos, el gravitino es inestable debido a la rotura de paridad R y puede decaer en partículas del Modelo Estándar a través de las interacciones bilineales que rompen paridad R. En este escenario, todos los efectos que violan paridad R, incluyendo el decaimiento del gravitino y las masas diferentes de cero de los neutrinos, son controlados por los acoplamientos bilineales $\xi_i$. 

La detección indirecta de materia oscura involucra las búsqueda de los estados finales en las desintegraciones del gravitino, entre los que se encuentran los rayos gamma. A partir de las restricciones sobre el tiempo de vida del gravitino se pueden obtener restricciones en función de la masa del gravitino para el acoplamiento bilineal dominante $\xi_3$. Cuando se asume la unificación de las masas de los gauginos a la escala de unificación, la restricción resultante sobre $\xi_3$ es muy fuerte: $\xi_3\lesssim 10^{-7}$ para $m_{\tilde G}\gtrsim 1$ GeV.  Sin embargo, la unificación de las masas de los gauginos no es una condición que se deba asumir necesariamente en los modelos supersimétricos, lo cual implicaría que la restricción sobre $\xi_3$ puede ser relajada.

Cuando los acoplamientos bilineales que violan paridad R son muy pequeños, las correspondientes valores para las masas y mezclas de los neutrinos generadas por dichos acoplamientos son negligibles. Sin embargo, aún se pueden explicar los datos de física de neutrinos a tráves de la implementación del mecanismo seesaw, agregando neutrinos derechos al contenido de partículas del Modelo Estándar supersimétrico. De hecho, los modelos con violación bilineal de paridad R con gravitino como materia oscura utilizan el mecanismo seesaw para explicar simultáneamente la física de neutrinos y la asimetría entre materia y antimateria por medio de la leptogénesis. En estos modelos se requiere que los acoplamientos bilineales sean muy pequeños $\xi_i<10^{-7}$ para no borrar la asimetría leptónica original, una temperatura de recalentamiento muy grande $T_R>10^9$ GeV y una masa para el gravitino mayor a 10 GeV. De estas restricciones se desprende el hecho que la leptogénesis excluye la posibilidad de generar las masas de los neutrinos por medio de los acoplamientos bilineales. No obstante, la posibilidad de generar las masas y mezclas de los neutrinos en modelos con violacion bilineal de paridad R y con el gravitino como materia oscura no quedaría excluida siempre y cuando la hipótesis de leptogénesis no se asuma.

En este proyecto lo que se pretende hacer es un estudio sistemático de los posibles modelos supersimétricos con violación bilineal de paridad R y con el gravitino como materia oscura, capaces de explicar la fenomenología de las oscilaciones de neutrinos sin tener en cuenta las restricciones teóricas impuestas por la leptogénesis  y por la universalidad de las masas de los gauginos. 

La partícula de materia oscura puede ser estable, o inestable con un
tiempo de vida mucho más grande que la edad del Universo. En el primer
caso usualmente se impone una simetría adicional $Z_2$ bajo la cual
las partículas del modelo estándar son pares, mientras que las
partículas nuevas son impares. Este tipo de simetría garantiza que la
partícula impar más liviana es estable y, de ser neutra, constituye un
buen candidato de materia oscura. En el programa computacional
micrOMEGAS~\cite{Belanger:2006is} se puede implementar cualquier
extensión del modelo estándar que posea una simetría $Z_2$ para
calcular numéricamente la densidad de reliquia de materia oscura y la
sección eficaz WIMP--nucleón, relevante para los experimentos de de
detección directa como XENON100. Cuando la partícula de materia oscura
es inestable se requiere de alguna simetría para garantizar que su
acoplamiento a las partículas del modelo estándar esté suficientemente
suprimido.

En el caso de supersimetría por ejemplo, todas las interacciones del
modelo están especificadas por la simetría gauge y el superpotencial,
el cual debe ser una función holomórfica de los campos escalares del
modelo. En el modelo estándar supersimétrico, además de los términos
del superpotencial que constituyen la semilla del potencial escalar y
del Lagrangiano de Yukawa, la invarianza gauge permite los siguientes
términos
\begin{equation}
  \label{eq:5}
  W_{\cancel{R_p}} = \mu_i\widehat{L}_i\widehat{H}_u + 
  \lambda_{i j k}\widehat{L}_i\widehat{L}_j\widehat{l}_k +
  \lambda'_{i j k}\widehat{L}_i\widehat{Q}_j\widehat{d}_k + 
  \lambda''_{ijk}\widehat{u}_i\widehat{d}_j\widehat{d}_k\,
\end{equation}
La simetría $Z_2$, conocida en éste caso como paridad R, $R_p$,
prohíbe todos estos términos, y cuando la partícula impar bajo paridad
R es el neutralino o el gravitino, ésta puede ser una candidato de
materia oscura. Aunque se rompa paridad R y alguno de estos términos
queden permitidos, el gravitino puede seguir siendo un candidato de
materia oscura pues sus decaimientos están suprimidos por la escala de
Planck. Para que el neutralino pueda ser materia oscura inestable
cuando se rompe paridad R, se requiere que los acoplamientos sean
extremadamente pequeños: del orden de $10^{-24}$ para los
acoplamientos trilineales ${\lambda^{(\ }}'\;''{}^)$. 

En modelos supersimétricos extendidos para incluir una simetría
horizontal anómala $U(1)_H$ a la Froggatt-Nielsen (FN)
\cite{Froggatt:1978nt}, las partículas del modelo estándar y sus
supercompañeros no llevan un número cuántico de paridad R, y en su
lugar llevan una carga horizontal (carga~$H$). Para una artículo de
revisión ver \cite{Dreiner:2003hw}.  Además, este tipo de modelos
involucran nuevos campos pesados de FN y, en la realización más
simple, un supercampo singlete electrodébil $\Phi$ de carga
$H=-1$. Términos efectivos invariantes bajo $SU(3)_c\times
SU(2)_L\times U(1)_Y\times U(1)_H$ que conservan y también que violan
paridad R, surgen una vez los grados de libertad pesados son
integrados debajo la escala de los campos FN ($M$), donde $M$
corresponde a la masa de Planck. Estos términos involucran factores
del tipo $(\Phi/M)^n$, donde $n$ esta fijado por las cargas
horizontales de los campos involucrados, y determina si un término en
particular puede o no estar presente en el superpotencial. La
holomorfía del superpotencial prohíbe todos los términos para los
cuales $n<0$, y aunque ellos se generaran después la ruptura de
supersimetría a través el potencial de K\"ahler \cite{Giudice:1988yz}
estos términos son en general mucho más suprimidos que aquellos para
los cuales $n\ge0$.  Términos con $n$ fraccionario están completamente
prohibidos. Finalmente, una vez $U(1)_H$ se rompe. los términos con $n$
positivo mantienen acoplamientos de Yukawa determinados, hasta
factores de orden uno, por $\theta^n=(\langle\phi\rangle/M)^n$. Los
acoplamientos de Yukawa del modelo estándar provienen típicamente de
términos de esta clase.

En este tipo de modelos, los ángulos de mezcla de los quarks, las masa
de los leptones cargados, y las condiciones de cancelación de
anomalías restringen los posibles asignamientos de cargas $H$
\cite{Leurer:1992wg,Binetruy:1996xk}. Ya que el número de
restricciones es siempre menor que el número de cargas $H$ algunas de
ellas quedan necesariamente sin restricciones, y aparte del límite
superior en sus valores pueden considerarse como parámetros libres que
se puede determinar a partir de información fenomenológica
adicional. 


En el caso de modelos supersimétricos basados una simetría de sabor
anómala $U(1)_H$, con un solo flavón de carga $-1$, una vez se
introducen las condiciones teóricas y fenomenológicas, la solución más
óptima se puede expresar en términos de 4 cargas $H$ libres que se
puede usar para explicar el valor de los 45 parámetros del
superpotencial que violan paridad R en la ec.~(\ref{eq:5})
\cite{Mira:2000gg,Dreiner:2003hw,Dreiner:2003yr,Dreiner:2007vp,Dreiner:2006xw,Sierra:2009zq}.

De las diferentes posibilidades se pueden construir modelos: con 
\begin{enumerate}
\item el modelo con ruptura bilineal de paridad R, es decir, con
  acoplamientos $\mu_i$ que explican las masas y mezclas de los
  neutrinos \cite{Mira:2000gg,Dreiner:2003hw,Dreiner:2006xw}.
\label{item:1}
\item modelos donde la simetría discreta de conservación de
  paridad R, o una simetría equivalente, queda como remanente de la
  ruptura espontánea de la simetría $U(1)_H$
  \cite{Dreiner:2003hw,Dreiner:2003yr,Dreiner:2007vp}. Las masas y
  mezclas de neutrinos se pueden explicar con la introducción de
  neutrinos derechos de carga $H$ semientera.
\label{item:2}
\item Modelos con violación de número leptónico a través de términos
  trilineales con ruptura de paridad R. Se pueden construir modelos
  con hasta dos términos del tipo $\lambda_{ijk}$, con los tres
  índices diferentes \cite{Sierra:2009zq}.  En este caso las cargas
  $H$ se pueden escoger de manera que los acoplamientos trilineales
  que violan paridad R queden muy suprimidos, del orden de $10^{-23}$,
  de modo que si los neutralinos son la partícula supersimétrica más
  livina (LSP de sus siglas en inglés), éstos pueden ser candidatos a
  materia oscura inestable con un tiempo de vida media del orden de
  $10^{26}\,$~sec. Dichos modelos se pueden usar para explicar las
  anomalías en rayos cósmicos \cite{Sierra:2009zq} recientemente
  detectadas por experimentos como PAMELA \cite{Adriani:2008zr}, ATIC
  \cite{:2008zzr} y Fermi~LAT \cite{Abdo:2009zk}. Estos datos sugieren
  que el decaimiento del neutralino sea principalmente leptónico, lo
  cual hace especialmente interesante que la única posibilidad
  consistente con simetrías horizontales de lugar precisamente a
  decaimientos leptónicos.
\label{item:3}
\item Modelos con violación de número bariónico a través de términos
  trilineales que violan $R$--paridad del tipo $\lambda''_{ijk}$. La
  fenomenología de este tipo de modelos la estámos desarrollando en un
  proyecto en marcha en el marco de la estrategia de Sostenibilidad
  del Grupo.
\label{item:4}
\end{enumerate}
En este proyecto pretendemos estudiar de manera sistemática la
fenomenología de los modelos \ref{item:1}. y \ref{item:3}. El caso
\ref{item:3}. sin embargo, nos enfocaremos en la posibilidad en la que
el LSP es el gravitino, donde creemos que también podemos explicar las
masas de los neutrinos, y además como en este caso los acoplamientos
trilineales pueden ser mucho más grandes, tenemos la posibilidad de
comprobar el modelo en el LHC.

%\subsection{Justificación del problema}

En el Grupo nos hemos enfocado en extensiones del modelo estándar que
dan cuenta de las masas y mezclas de neutrinos. Hemos estudiado
exhaustivamente las predicciones del modelo con ruptura bilineal de
paridad R que incluye sólo los tres términos con $\mu_i$ (caso
\ref{item:1}.), tanto para el Tevatron como para el LHC, asumiendo que
el neutralino es la LSP
\cite{Magro:2003zb,deCampos:2005ri,deCampos:2007bn,deCampos:2008ic,deCampos:2008re,DeCampos:2010yu}. En
el último trabajo al respecto hemos determinado el nivel de precisión
con el que se puede llegar a medir en el LHC la correlación entre
decaimientos de neutralinos a muón y tau con el ángulo de mezcla
atmosférico de neutrinos: una predicción muy concreta que de no
observarse en el LHC en los próximos años descartaría completamente el
modelo como el mecanismo de generación de masas para neutrinos.
\begin{proyecto}
  En este proyecto pretendemos seguir explorando más correlaciones de
  de observables en el LHC con física de neutrinos para determinar con
  que nivel de precisión se podrían llegar a medir en el LHC. Cuando
  el neutralino es la LSP, los decaimientos a tres cuerpos mediados
  por sfermiones con muones y electrones en los estados finales, están
  correlaciones con el ángulo de mezcla solar, y la longitud de
  decaimiento del neutralino está correlacionada con la diferencia de
  masa atmosférica.
\end{proyecto}

Cuando el gravitino es la LSP, se constituye en un candidato viable de
materia oscura inestable. Éste caso ha sido estudiado exhaustivamente
en la literatura por sus implicaciones en experimentos de rayos
cósmicos, especialmente en el caso en que las masas de neutrinos son
explicadas a través del mecanismo de seesaw. En tal caso el modelo
también explica bariogénisis a través de leptogénesis. De hecho, como
hemos mostramos en \cite{Choi:2010jt}, y ratificado por otro grupo con
datos más recientes en \cite{Garny:2010eg}, los últimos datos de
líneas de rayos gamma publicado por FERMI-LAT excluyen la posibilidad
de que las masas de neutrinos puedan ser generadas por términos
bilineales de ruptura de paridad R si la temperatura de
recalentamiento está sobre $10^9\ $GeV como sugiere leptogénesis. En
esta región del espacio de parámetros el gravitino es mayor de unos 10
GeV y los acoplamientos bilineales son tan pequeños que el decaimiento
de la partícula siguiente a la LSP, la NLSP (de sus siglas en inglés),
ocurre fuera del detector en el LHC, por lo que el modelo es
básicamente indistinguible del MSSM. En éste caso los datos de
detección indirecta de materia oscura a través de rayos cósmicos
serían la única forma de diferenciar el modelo del MSSM (donde el
neutralino es el candidato de materia oscura). Sin embargo, si
consideremos masas de gravitinos más pequeñas, aunque ya no podríamos
explicar bariogénesis a través de leptogénesis, recuperamos la
posibilidad de explicar las masas y mezclas de los neutrinos a través de
los términos bilineales que rompen paridad R~\cite{Hirsch:2005ag}, un
mecanismo, que a diferencia del seesaw, si se puede verificar en el
LHC.

\begin{proyecto}
  En este proyecto realizaremos las simulaciones de las señales del
  NSLP para el LHC  en las regiones del
  espacio de parámetros donde el gravitino es un buen candidato a
  materia oscura y los términos bilineales explican los datos de
  oscilaciones de neutrinos. 
\end{proyecto}


La otra posibilidad de tener términos de violación de número leptónico
en el superpotencial compatible con la simetría horizontal, es el caso
\ref{item:3}. donde tenemos un modelo con ruptura trilineal de paridad
R a través de términos $\lambda$. Esta posibilidad es muy llamativa
porque da lugar a un candidato de materia oscura que decae sólo
leptónicamente. Nuestro grupo \cite{Nardi:2008ix} fue uno de los
primeros en proponer una explicación en términos de materia oscura
inestable para explicar el exceso de positrones observado por el
satelite PAMELA en el 2008 \cite{Adriani:2008zr}. Luego hemos
construido un modelo basado en supersimetría con ruptura de paridad R
a través de términos trilineales del tipo $\lambda$, caso
\ref{item:3}, para explicar la preferencia por decaimientos leptónicos
de la partícula de materia oscura, que en este caso es el
neutralino \cite{Sierra:2009zq}.
\begin{proyecto}
  En este proyecto queremos explorar la posibilidad de explicar
  también las masas de neutrinos, teniendo al gravitino como materia
  oscura.
\end{proyecto}
En un proyecto en marcha estamos explorando la posibilidad opuesta
\ref{item:4}., de un candidato de materia oscura que decae sólo
hadrónicamente. La simetría horizontal garantiza que sólo los
acoplamientos trilineales de quarks derechos estén permitidos. En
dicho proyecto hemos extendido el modelo para incluir masas de
neutrinos. Ésta experiencia nos servirá para hacer las modificaciones
necesarias al modelo con acoplamientos trilineales leptónicos del tipo
$\lambda$ para dar cuenta de masas y mezclas para los neutrinos.


\begin{darkmatter}
  La metodología desarrollada en el estudio exhaustivo del modelo con
  ruptura bilineal de paridad R la hemos logrado aplicar a otros
  modelos de generación radiativa de masas de neutrinos con nuevas
  partículas a la escala del TeV asequibles en el LHC
  \cite{Sierra:2008wj,AristizabalSierra:2006ri}. Recientemente hemos
  comenzado a explorar modelos que puedan dar cuenta simultáneamente
  de masas de neutrinos y candidatos de materia oscura
  \cite{Hirsch:2005ag,Choi:2010jt,Sierra:2008wj}. En el caso del
  seesaw radiativo, donde la violación de número leptónico se da a
  nivel del potentcial escalar, se implementa una simetría $Z_2$ de
  manera que la masa de neutrinos sólo pueda ser generada
  radiativamente con los neutrinos derechos a la escala del TeV
  garantizando que el modelo pueda comprobarse en el LHC. La misma
  simetría garantiza que el escalar o un neutrino derecho puedan ser
  buenos candidatos de materia oscura. Finalmente, la estructura
  escalar del modelo permite tener el Higgs mucho más pesado que en el
  modelo estándar de forma compatible con las correcciones
  electrodébiles.
% Las regiones del modelo en la cual la partícula de materia oscura es
% escalar ha sido bastante estudiada en la literatura en el contexto
% del modelo conocido como doblete inerte \cite{LopezHonorez:2010tb}. Sin
% embargo, las regiones en las cuales el neutrino derecho es el
% candidato de materia oscura todavía requieren de


\begin{proyecto}
  Todo esto hace muy llamativo el estudio de las señales del seesaw
  radiativo para el LHC, y en los experimentos de detección directa de
  materia oscura. En este proyecto pretendemos establecer todas las
  regiones del espacio de parámetros donde se puede tener la densidad
  de reliquia de materia oscura y las masas y mezclas de neutrinos
  adecuadas. En cada región se establecerán las señales que se esperan
  en el LHC y en experimentos de detección directa de materia oscura,
  y para las señales más representativas hacer simulaciones para el
  detector ATLAS del LHC.
\end{proyecto}
\end{darkmatter}


\begin{proyecto}
  En términos generales, en este proyecto queremos explorar modelos
  motivados por falencias fenomenológicas del modelo estándar,
  especialmente aquellos modelos que no sólo tienen implicaciones en
  el LHC, sino también en experimentos de detección de rayos cósmicos,
  y experimentos de detección directa de materia oscura.
\end{proyecto}


\subsection{Objetivos:                                     }
\subsection{Metodología:                                   }
\subsection{Actividades:                                   }
\subsection{Resultados esperados:                          }
\subsection{Cronograma:                                    }

\subsection{Equipo de trabajo:}
\begin{tabular}{|l|l|l|l|}\hline
Recurso humano (rol)& Responsabilidad& Unidades (días, meses)& \# de unidades\\\hline
&&&\\\hline
\end{tabular}

\subsection{Presupuesto}
\begin{instrucciones}
  Por favor siga el modelo de la tabla que encontrará en este punto para la presentación del presupuesto.
\end{instrucciones}
\begin{tabular}{|l|l|l|l|}\hline
  \multirow{2}{*}{Rubros}&\multicolumn{2}{c}{Fuentes}\vline&\multirow{2}{*}{Total}\\
  \cline{2-3} & Colciencias & Contrapartida & \\\hline 
 & & &\\\hline
 & & &\\\hline
 & & &\\\hline
 & & &\\\hline
TOTAL (en pesos) & & &\\\hline
Total en porcentaje & & &\\\hline
\end{tabular}

\subsection{Plan de acción.}
\begin{tabular}{|l|l|l|l|l|l|}\hline
  \multirow{2}{*}{Objetivo} & \multirow{2}{*}{Estrategia} & \multirow{2}{*}{Indicador}  & \multicolumn{3}{|c|}{Metas}\\
\cline{4-6} 
& & & Bimestre 1 &Bimestre 2 & Bimestre 3\\\hline 
&&&&&\\\hline
\end{tabular}



%%% Local Variables: 
%%% mode: latex
%%% TeX-master: "proyecto"
%%% End: 
