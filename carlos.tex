\section{Información sobre el colombiano en el exterior}
\subsection{Nombres y apellidos: }
Carlos Esteban Yaguna Toro
\subsection{Lugar y fecha de nacimiento: }
\fnyaguna
\subsection{Correo electrónico: }
\emyaguna
\subsection{País y ciudad de residencia: }
Alemania, M\"unster
\subsection{Formación académica:}
\subsection{Producción intelectual de los tres últimos años}
\subsubsection{Artículos publicados en revistas indexadas:}
\begin{itemize}
\item   D.~Aristizábal Sierra, C.~E.~Yaguna,
  ``On the importance of the 1-loop finite corrections to seesaw neutrino masses,''
  JHEP {\bf 1108 } (2011)  013.

\item   C.~E.~Yaguna,
  ``The singlet scalar as FIMP dark matter,''
  JHEP {\bf 1108 } (2011)  060.

\item   L.~López Honorez, C.~E.~Yaguna,
  ``A new viable region of the inert doublet model,''
  JCAP {\bf 1101 } (2011)  002.

\item   K.~-Y.~Choi, D.~Restrepo, C.~E.~Yaguna, O.~Zapata,
  ``Indirect detection of gravitino dark matter including its three-body decays,''
  JCAP {\bf 1010 } (2010)  033.

\item   S.~Hummer, M.~Maltoni, W.~Winter, C.~Yaguna,
  ``Energy dependent neutrino flavor ratios from cosmic accelerators on the Hillas plot,''
  Astropart.\ Phys.\  {\bf 34 } (2010)  205-224.

\item   K.~-Y.~Choi, C.~E.~Yaguna,
  ``New decay modes of gravitino dark matter,''
  Phys.\ Rev.\  {\bf D82 } (2010)  015008.

\item   L.~López Honorez, C.~E.~Yaguna,
  ``The inert doublet model of dark matter revisited,''
  JHEP {\bf 1009 } (2010)  046.

\item   C.~E.~Yaguna,
  ``Large contributions to dark matter annihilation from three-body final states,''
  Phys.\ Rev.\  {\bf D81 } (2010)  075024.

\item   C.~E.~Yaguna,
  ``Gamma ray lines: What will they tell us about SUSY?,''
  Phys.\ Rev.\  {\bf D80 } (2009)  115002.

\item   A.~Goudelis, Y.~Mambrini, C.~Yaguna,
  ``Antimatter signals of singlet scalar dark matter,''
  JCAP {\bf 0912 } (2009)  008.

\item   K.~-Y.~Choi, C.~E.~Yaguna,
  ``Implications of an astrophysical interpretation of PAMELA and Fermi-LAT data for future searches of a positron signal from dark matter annihilations,''
  Phys.\ Rev.\  {\bf D81 } (2010)  023502.

\end{itemize}
\subsubsection{Libros y capítulos publicados:}
\subsection{Reconocimiento por excelencia académica:}
%maximo 200 palabras
\subsection{Experiencia específica en el tema del proyecto:}
%maximo 200 palabras
El Dr. Yaguna es un reconocido experto en extensiones del ME para
explicar la materia oscura del Universo, en especial en modelos
supersimétricos donde cuenta con unos 20 artículos publicados. En
particular es un experto en el uso de MicroMEGAs, un programa para
calcular la densidad de reliquia y la sección eficaz WIMP--nucleón en
extensiones del ME basadas en una simetría $Z_2$ con un candidato de
materia oscura~\cite{Yaguna:2011qn,LopezHonorez:2010tb,Honorez:2010re,Choi:2009qc,Yaguna:2008hd,Goudelis:2009zz}. También tiene experiencia en modelos con rotura
bilineal de paridad con el gravitino como materia oscura~\cite{Choi:2010jt,Choi:2010xn} y en modelo con rotura trilineal de paridad R~\cite{AristizabalSierra:2008ye}. 

Con el Dr. Aristizábal y el Dr. Yaguna hemos realizado un trabajado
sobre decaimientos exóticos del Higgs en modelos con rotura de paridad
R~\cite{AristizabalSierra:2008ye}.  La colaboración del Dr. Yaguna nos
ha permitido abrir una fructífera línea de investigación sobre
decaimientos de gravitino en modelos con violación de paridad
R~\cite{Choi:2010jt}.
\subsection{Trayectoria laboral:}
\begin{instrucciones}
  Liste su experiencia laboral en orden cronológico (empezando por lo
  actual) organizándolos cada referencia por los siguientes criterios:
  nombre de la empresa o entidad, pública o privada, país, cargo,
  fecha de ingreso y fecha de retiro.
\end{instrucciones}
\begin{itemize}
\item M\"unster University, Alemania. Investigador posdoctoral. October 2011-present

\item W\"urzburg University, Alemania. Investigador posdoctoral. March 2011-September 2011

\item Universidad Autónoma de Madrid, España.Investigador posdoctoral. April 2008-February 2011

\item Universidad de California (Los Angeles), USA. Investigador posdoctoral. November 2004-October 2007, SISSA-ISAS, Italia. 

\item Estudiante de doctorado. November 2001-October 2004.

\end{itemize}
\subsection{Tiempo total de experiencia:}
\begin{instrucciones}
  Indique el tiempo total de su experiencia laboral en número de años
  y meses (ejemplo: 5 años, 3 meses).
\end{instrucciones}
10 años.

%%% Local Variables: 
%%% mode: latex
%%% TeX-master: "proyecto"
%%% End: 
