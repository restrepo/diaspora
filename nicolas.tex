\section{Información sobre el colombiano en el exterior}
\subsection{Nombres y apellidos: }
Nicolás Bernal Hernández
\subsection{Lugar y fecha de nacimiento: }
\fnbernal
\subsection{Correo electrónico: }
\embernal
\subsection{País y ciudad de residencia: }
Alemania, Bonn
\subsection{Formación académica:}
\begin{itemize}
\item Físico, Universidad del Valle (Colombia).
\item M. Sc.,  Université Pierre et Marie Curie, Paris 6 (Francia).
\item Ph. D.,  Université Pierre et Marie Curie, Paris 6 (Francia).
\end{itemize}
\subsection{Producción intelectual de los tres últimos años}
\subsubsection{Artículos publicados en revistas indexadas:}
\begin{itemize}
\item   N.~Bernal, M.~Losada, F.~Mahmoudi,
  ``Flavour physics constraints in the BMSSM,''
  JHEP {\bf 1107 } (2011)  074.

\item   N.~Bernal,
  ``Reconstructing Dark Matter Properties via Gamma-Rays with Fermi-LAT,''
  PoS {\bf IDM2010 } (2011)  022.

\item   D.~López-Val, J.~Solà, N.~Bernal,
  ``Quantum effects on Higgs-strahlung events at Linear Colliders within the general 2HDM,''
  Phys.\ Rev.\  {\bf D81 } (2010)  113005.

\item   N.~Bernal, A.~Goudelis,
  ``Dark matter detection in the BMSSM,''
  JCAP {\bf 1003 } (2010)  007.

\item   N.~Bernal, K.~Blum, Y. Nir, M.~Losada,
  ``BMSSM Implications for Cosmology,''
  JHEP {\bf 0908 } (2009)  053.

\item   N.~Bernal,
  ``Dark matter direct detection in the MSSM with heavy scalars,''
  JCAP {\bf 0908 } (2009)  022.

\item   N.~Bernal, D.~López-Val, J.~Solà,
  ``Single Higgs-boson production through gamma-gamma scattering within the general 2HDM,''
  Phys.\ Lett.\  {\bf B677 } (2009)  39-47.

\item   N.~Bernal, A.~Goudelis, Y.~Mambrini, C.~Muñoz,
  ``Determining the WIMP mass using the complementarity between direct and indirect searches and the ILC,''
  JCAP {\bf 0901 } (2009)  046.

\end{itemize}
\subsubsection{Libros y capítulos publicados:}
\subsection{Reconocimiento por excelencia académica:}
%maximo 200 palabras
\subsection{Experiencia específica en el tema del proyecto:}
%maximo 200 palabras
El Dr. Bernal es un experto en modelos supersimétricos, en particular
en programas para generar espectros supersimétricos~\cite{Bernal:2007uv}, en señales de
aniquilación de matera oscura estable en experimentos de rayos
cósmicos~\cite{Bernal:2011pz,Bernal:2010ip,Bernal:2008zk}, y en señales en experimentos de detección directa de materia
oscura~\cite{Bernal:2009tt}. Es también un experto en extensiones del MSSM con términos no
renormalizables de dimensión cinco~\cite{Bernal:2011pj,Bernal:2010uf,Bernal:2009jc,Bernal:2009hd}.

\subsection{Trayectoria laboral:}
\begin{itemize}
\item Physikalisches Institut y Bethe Center for Theoretical Physics, Universität Bonn, entidad pública, Alemania, Postdoc, Octubre 2010 -- Septiembre 2013.
\item Centro de Física Teórica de Partículas do Instituto Superior Técnico, Universidade Técnica de Lisboa, entidad pública, Portugal, Postdoc, Octubre 2009 -- Septiembre 2010.
\item Departament d'Estructura i Constituents de la Matèria e Institut de Ciències del Cosmos, Universitat de Barcelona, entidad pública, España, Postdoc, Octubre 2008 -- Septiembre 2009.
\item Laboratoire de Physique Théorique, Université Pierre et Marie Curie, entidad pública, Francia, Doctorante, Octubre 2005 -- Septiembre 2008.
\end{itemize}
\subsection{Tiempo total de experiencia:}
6 años.


%%% Local Variables: 
%%% mode: latex
%%% TeX-master: "proyecto"
%%% End: 
